\documentclass[10pt,]{article}
\usepackage[margin=1in]{geometry}
\newcommand*{\authorfont}{\fontfamily{phv}\selectfont}

\usepackage{xcolor}
\definecolor{ttcolor}{RGB}{255,40,40}

\newenvironment{cslreferences}%
  {}%
  {\par}

% redefinition of \texttt
\let\Oldtexttt\texttt
\renewcommand\texttt[1]{{\ttfamily\color{ttcolor}#1}}

\usepackage{caption}
\captionsetup{font={stretch=1.2,small},
              labelfont=bf,
              textfont=it,
              margin=0.5cm,
              width=.90\textwidth}
% captions
% Figure names
\renewcommand{\figurename}{Fig.}

\usepackage{tikz}
\usetikzlibrary{trees}

\usepackage{abstract}
\renewcommand{\abstractname}{}    % clear the title
\renewcommand{\absnamepos}{empty} % originally center
\newcommand{\blankline}{\quad\pagebreak[2]}

\usepackage{amsmath}
\usepackage{algorithm}
\usepackage[noend]{algpseudocode}

\makeatletter
\def\BState{\State\hskip-\ALG@thistlm}
\makeatother

\usepackage{mathspec}

\usepackage{fontspec}
\setmonofont[Scale=MatchLowercase]{JetBrains Mono Regular}
\setallmainfonts(Digits,Latin){XCharter}
\setmainfont{XCharter}

\providecommand{\tightlist}{%
  \setlength{\itemsep}{0pt}\setlength{\parskip}{0pt}} 
\usepackage{longtable,booktabs}

\usepackage{titlesec}
\titlespacing\section{0pt}{12pt plus 4pt minus 2pt}{6pt plus 2pt minus 2pt}
\titlespacing\subsection{0pt}{12pt plus 4pt minus 2pt}{6pt plus 2pt minus 2pt}
\titleformat*{\subsection}{\bfseries\itshape}
\titleformat{\section}{\normalfont\fontsize{12}{15}\bfseries}{\thesection}{1em}{}

\usepackage{titling}
\setlength{\droptitle}{-.25cm}

% better paragraph indents
\edef\restoreparindent{\parindent=\the\parindent\relax}
\usepackage{parskip}
\restoreparindent

\linespread{1.05}

\usepackage{fancyhdr}
\pagestyle{fancy}
\usepackage{lastpage}
\renewcommand{\headrulewidth}{0.3pt}
\renewcommand{\footrulewidth}{0.0pt} 
\lhead{\footnotesize \textbf{201374125}}
\chead{}
\rhead{\footnotesize \emph{Building a Graph Database of Place
Descriptions}}
%\lfoot{}
%\cfoot{\small \thepage/\pageref*{LastPage}}
%\rfoot{}

\fancypagestyle{firststyle}
{
\renewcommand{\headrulewidth}{0pt}%
   \fancyhf{}
   \fancyfoot[C]{\small \thepage/\pageref*{LastPage}}
}

%\def\labelitemi{--}
%\usepackage{enumitem}
%\setitemize[0]{leftmargin=25pt}
%\setenumerate[0]{leftmargin=25pt}


\usepackage{titlesec}


\makeatletter
\@ifpackageloaded{hyperref}{}{%
\ifxetex
  \usepackage[setpagesize=false, % page size defined by xetex
              unicode=false, % unicode breaks when used with xetex
              xetex]{hyperref}
\else
  \usepackage[unicode=true]{hyperref}
\fi
}
\@ifpackageloaded{color}{
    \PassOptionsToPackage{usenames,dvipsnames}{color}
}{%
    \usepackage[usenames,dvipsnames]{color}
}
\makeatother
\hypersetup{breaklinks=true,
            bookmarks=true,
            pdfauthor={ ()},
             pdfkeywords = {},  
            pdftitle={Building a Graph Database of Place Descriptions},
            colorlinks=true,
            citecolor=blue,
            urlcolor=blue,
            linkcolor=magenta,
            pdfborder={0 0 0}}
\urlstyle{same}  % don't use monospace font for urls


\setcounter{secnumdepth}{0}

\usepackage{color}
\usepackage{fancyvrb}
\newcommand{\VerbBar}{|}
\newcommand{\VERB}{\Verb[commandchars=\\\{\}]}
\DefineVerbatimEnvironment{Highlighting}{Verbatim}{commandchars=\\\{\}}
% Add ',fontsize=\small' for more characters per line
\usepackage{framed}
\definecolor{shadecolor}{RGB}{248,248,248}
\newenvironment{Shaded}{\begin{snugshade}}{\end{snugshade}}
\newcommand{\AlertTok}[1]{\textcolor[rgb]{0.94,0.16,0.16}{#1}}
\newcommand{\AnnotationTok}[1]{\textcolor[rgb]{0.56,0.35,0.01}{\textbf{\textit{#1}}}}
\newcommand{\AttributeTok}[1]{\textcolor[rgb]{0.77,0.63,0.00}{#1}}
\newcommand{\BaseNTok}[1]{\textcolor[rgb]{0.00,0.00,0.81}{#1}}
\newcommand{\BuiltInTok}[1]{#1}
\newcommand{\CharTok}[1]{\textcolor[rgb]{0.31,0.60,0.02}{#1}}
\newcommand{\CommentTok}[1]{\textcolor[rgb]{0.56,0.35,0.01}{\textit{#1}}}
\newcommand{\CommentVarTok}[1]{\textcolor[rgb]{0.56,0.35,0.01}{\textbf{\textit{#1}}}}
\newcommand{\ConstantTok}[1]{\textcolor[rgb]{0.00,0.00,0.00}{#1}}
\newcommand{\ControlFlowTok}[1]{\textcolor[rgb]{0.13,0.29,0.53}{\textbf{#1}}}
\newcommand{\DataTypeTok}[1]{\textcolor[rgb]{0.13,0.29,0.53}{#1}}
\newcommand{\DecValTok}[1]{\textcolor[rgb]{0.00,0.00,0.81}{#1}}
\newcommand{\DocumentationTok}[1]{\textcolor[rgb]{0.56,0.35,0.01}{\textbf{\textit{#1}}}}
\newcommand{\ErrorTok}[1]{\textcolor[rgb]{0.64,0.00,0.00}{\textbf{#1}}}
\newcommand{\ExtensionTok}[1]{#1}
\newcommand{\FloatTok}[1]{\textcolor[rgb]{0.00,0.00,0.81}{#1}}
\newcommand{\FunctionTok}[1]{\textcolor[rgb]{0.00,0.00,0.00}{#1}}
\newcommand{\ImportTok}[1]{#1}
\newcommand{\InformationTok}[1]{\textcolor[rgb]{0.56,0.35,0.01}{\textbf{\textit{#1}}}}
\newcommand{\KeywordTok}[1]{\textcolor[rgb]{0.13,0.29,0.53}{\textbf{#1}}}
\newcommand{\NormalTok}[1]{#1}
\newcommand{\OperatorTok}[1]{\textcolor[rgb]{0.81,0.36,0.00}{\textbf{#1}}}
\newcommand{\OtherTok}[1]{\textcolor[rgb]{0.56,0.35,0.01}{#1}}
\newcommand{\PreprocessorTok}[1]{\textcolor[rgb]{0.56,0.35,0.01}{\textit{#1}}}
\newcommand{\RegionMarkerTok}[1]{#1}
\newcommand{\SpecialCharTok}[1]{\textcolor[rgb]{0.00,0.00,0.00}{#1}}
\newcommand{\SpecialStringTok}[1]{\textcolor[rgb]{0.31,0.60,0.02}{#1}}
\newcommand{\StringTok}[1]{\textcolor[rgb]{0.31,0.60,0.02}{#1}}
\newcommand{\VariableTok}[1]{\textcolor[rgb]{0.00,0.00,0.00}{#1}}
\newcommand{\VerbatimStringTok}[1]{\textcolor[rgb]{0.31,0.60,0.02}{#1}}
\newcommand{\WarningTok}[1]{\textcolor[rgb]{0.56,0.35,0.01}{\textbf{\textit{#1}}}}




\usepackage{setspace}

\title{Building a Graph Database of Place Descriptions}
\author{201374125}
\date{}


\def\citeapos#1{\citeauthor{#1}'s (\citeyear{#1})}

% header includes!
\linespread{1.05}


\begin{document}  



\thispagestyle{plain} 

\begin{flushleft}\Large \bf Building a Graph Database of Place
Descriptions  \end{flushleft}
	\vspace{1 mm}   
201374125 \\
\emph{} \\
    \\

% \blankline 
  

\hrule

\vspace{6 mm}
	


\hypertarget{introduction}{%
\section{Introduction}\label{introduction}}

This project aims to provide a structured graph database of location
descriptions for use in geographic natural language processing tasks.
Using a graph database for the store of this information allows for
semantic associations between locations to be inferred based on their
proximity within the database. This provides an alternative to the
purely coordinate based and euclidean associations between geographic
locations, and attempts to capture platial connections, as opposed to
purely spatial ones. This considers previous work addressing the
development of hierarchical geo-ontologies (Sun \emph{et al.},
\protect\hyperlink{ref-sun2019a}{2019}), and attempts to describe the
hierarchical nature of definable geographic concepts in a computer
interpretable way, demonstrated on \textbf{Fig. \ref{f:hier}}.

\begin{figure}[H]
\begin{center}
\begin{tikzpicture}[level distance=1.5cm,
  level 1/.style={sibling distance=3cm},
  level 2/.style={sibling distance=1.5cm}]
  \node {Country}
    child {node {Region}}
    child {node {Region}
        child {node {County}}
        child {node {County}
          child {node {District}}
          child {node {District}
              child {node {Postcode}
                child {node {Location}}}
      }}
    };
\end{tikzpicture}
\end{center}
\caption{The hierarchical representation of geographical concepts}\label{f:hier}
\end{figure}

Purely geographic implementations of similar work already exist, for
example \href{https://geonames.org}{Geonames}, and
\href{https://openstreetmap.org}{OpenStreetMap}, which both have
comprehensive coverage of places for much of the UK. However, these do
not provide any descriptive language associated with the geographic
locations. \href{http://dbpedia.org}{DBpedia} is an example of a more
general knowledge base, and given it is built from data extracted from
Wikipedia, it provides descriptions for many of the items contained
within it.

Labelled data is often a primary concern when performing many analytical
tasks, and is considered a particular issue in geographic natural
language processing (Gritta, \protect\hyperlink{ref-gritta2019}{2019};
Stock \emph{et al.}, \protect\hyperlink{ref-stock2013}{2013}). Much of
the existing work has relied on time consuming, manual labelling of data
which leads to smaller datasets (Middleton \& Krivcovs,
\protect\hyperlink{ref-middleton2016}{2016}; Wallgrün \emph{et al.},
\protect\hyperlink{ref-wallgrun2018}{2018}; Gey \emph{et al.},
\protect\hyperlink{ref-gey2006}{2006}), and many which are not made
freely available (Leidner \& Lieberman,
\protect\hyperlink{ref-leidner2011}{2011}; Leidner,
\protect\hyperlink{ref-leidner2007}{2007}; Andogah,
\protect\hyperlink{ref-andogah2010}{2010}; Weissenbacher \emph{et al.},
\protect\hyperlink{ref-weissenbacher2019}{2019}). The lack of large,
high quality, labelled geographic natural language data is well noted by
many authors in this subject area (Tobin \emph{et al.},
\protect\hyperlink{ref-tobin2010}{2010}; Speriosu,
\protect\hyperlink{ref-speriosu2013}{2013}; Weissenbacher \emph{et al.},
\protect\hyperlink{ref-weissenbacher2015}{2015},
\protect\hyperlink{ref-weissenbacher2019}{2019}; Gritta \emph{et al.},
\protect\hyperlink{ref-gritta2018}{2018}; Karimzadeh \emph{et al.},
\protect\hyperlink{ref-karimzadeh2019}{2019}).

The production of a dataset containing descriptive information, in
addition to hierarchical geo-information regarding locations may assist
with a variety of geographic natural language tasks. For example, in
toponym disambiguation, contextual information provided alongside the
identified toponym is often used as a method for correctly resolving to
a single toponym (Tobin \emph{et al.},
\protect\hyperlink{ref-tobin2010}{2010}; Roberts,
\protect\hyperlink{ref-roberts2010}{2010}; Speriosu,
\protect\hyperlink{ref-speriosu2013}{2013}), this may include topics
associated with particular toponyms (Speriosu,
\protect\hyperlink{ref-speriosu2013}{2013}; Adams \& McKenzie,
\protect\hyperlink{ref-adams2013}{2013}; Ju \emph{et al.},
\protect\hyperlink{ref-ju2016}{2016}), and metadata associated with the
toponyms, including geotags (Zhang \& Gelernter,
\protect\hyperlink{ref-zhang2014}{2014}), and other structured
information (Weissenbacher \emph{et al.},
\protect\hyperlink{ref-weissenbacher2015}{2015}). Additionally, this
dataset acts as labelled descriptive information regarding a specific,
known (geocoded) location, useful for recent developments in
fine-grained localisation research (Al-Olimat \emph{et al.},
\protect\hyperlink{ref-al-olimat2019}{2019}; Chen \emph{et al.},
\protect\hyperlink{ref-chen2018a}{2018}\protect\hyperlink{ref-chen2018a}{a},
\protect\hyperlink{ref-chen2018}{2018}\protect\hyperlink{ref-chen2018}{b}),
and point of interest identification (Moncla \emph{et al.},
\protect\hyperlink{ref-moncla2014}{2014}; Li \& Sun,
\protect\hyperlink{ref-li2014}{2014}).

This project therefore aims to bring forward the most complete corpus of
labelled geographic natural language available through automatic
extraction of Wikipedia place summaries, providing both contextual
information associated with toponyms and hierarchical geo-information.

\hypertarget{methodology}{%
\section{Methodology}\label{methodology}}

\hypertarget{dbpedia}{%
\subsection{DBpedia}\label{dbpedia}}

\href{https://wiki.dbpedia.org/}{DBpedia} is a crowd sourced collection
of information extracted from
\href{https://commons.wikimedia.org/wiki/Main_Page}{Wikimedia} projects,
presented in a structured format resembling an open knowledge graph
(OKG). This provides linked data in a machine-readable format,
accessible through a \href{http://www.dbpedia.org/sparql}{SPARQL
querying API}. The data follows the
\href{https://www.w3.org/RDF/}{Resource Description Framework} (RDF) as
defined by the World Wide Web Consortium (W3C) specifications, providing
an alternative web linking structure. RDF models for data exchange use
URIs to name a relationship between things, as well as information
regarding the two ends of each link, generally known as a triple.

As of this report, the DBpedia knowledge base describes 4.58 million
things, including persons, places, creative works, and organisations.
The data is available under the
\href{https://creativecommons.org/licenses/by-sa/3.0/}{Creative Commons
Attribution-ShareAlike 3.0 Licence} and the
\href{https://www.gnu.org/licenses/fdl-1.3.en.html}{GNU Free
Documentation Licence} which allows for copying, redistribution and
adaptation of the data, including for commercial use.

\hypertarget{building-a-sparql-query}{%
\subsection{Building a SPARQL Query}\label{building-a-sparql-query}}

SPARQL is an RDF query language which allows for the use of namespace
prefixes to query URI triples from the DBpedia RDF database. These
prefixes include DBpedia defined
\href{http://dbpedia.org/ontology/}{ontologies},
\href{http://dbpedia.org/resource/}{resources}, or
\href{http://dbpedia.org/property/}{properties}, and additional
prefixes, including those defined by the W3C. The DBpedia prefixes are
perhaps the most useful, as they provide consistent class definitions
for types of thing contained in the database. For example anything given
the class \texttt{Place} is likely of interest for this research. This
class \texttt{Place} then provides various subclasses which may be used
to perform specific queries, \texttt{Architectural\ Structure} and
Celestial \texttt{Body} are examples of \texttt{Place} subclasses. These
subclasses also provide their own subclasses, e.g.~\texttt{Building},
\texttt{Pyramid.}

However, from inspection of specific URIs for places within the United
Kingdom, it appears that these RDF links are often incomplete, or
provide mismatched information. It is rare for a place that is not
classified as a \texttt{Building} to have \texttt{Building} level class
granularity, and many places in the UK only contain the \texttt{Place}
class. Additionally, the \texttt{Country} relation for many places
returns primarily the \emph{United Kingdom}, but includes counties
e.g.~\emph{Dorset}, and occasionally returns \emph{England}, but never
\emph{Scotland} or \emph{Wales}. The same is true with
\texttt{District}, which occasionally returns cities
e.g.~\emph{Burthwaite} returns \emph{City of Carlisle}. Some naming is
inconsistent e.g.~\emph{Lewes (district)} vs \emph{Scarborough
(borough)} and \emph{South Lakeland} are all labels given to
\texttt{Place} in the United Kingdom, chiefly when place names are not
unique they contain additional information in parentheses. Therefore, it
is unlikely that comprehensive coverage may be achieved if attempting to
query places using their more granular classes. It is also important for
the inclusion of all places that, instead of just querying using the
\emph{United Kingdom}, the query includes the constituent countries.
Coordinate information is also present for many, but not all, DBpedia
places.

Considering the limitations of the database in relation to the United
Kingdom, a query was built to obtain English Wikipedia abstracts for
each \texttt{Place} in any country within Great Britain, using the
\href{http://dbpedia.org/sparql}{DBpedia SPARQL endpoint}. Given the
above constraints, no further information was extracted. This query is
given below:

\vspace{2mm}

\begin{Shaded}
\begin{Highlighting}[]
\NormalTok{    PREFIX dbo: }\OperatorTok{\textless{}}\NormalTok{http:}\OperatorTok{//}\NormalTok{dbpedia.org}\OperatorTok{/}\NormalTok{ontology}\OperatorTok{/\textgreater{}}
\NormalTok{    PREFIX res: }\OperatorTok{\textless{}}\NormalTok{http:}\OperatorTok{//}\NormalTok{dbpedia.org}\OperatorTok{/}\KeywordTok{resource}\OperatorTok{/\textgreater{}}
\NormalTok{    PREFIX rdf: }\OperatorTok{\textless{}}\NormalTok{http:}\OperatorTok{//}\NormalTok{www.w3.org}\OperatorTok{/}\DecValTok{1999}\OperatorTok{/}\DecValTok{02}\OperatorTok{/}\DecValTok{22}\OperatorTok{{-}}\NormalTok{rdf}\OperatorTok{{-}}\NormalTok{syntax}\OperatorTok{{-}}\NormalTok{ns\#}\OperatorTok{\textgreater{}}
\NormalTok{    PREFIX rdfs: }\OperatorTok{\textless{}}\NormalTok{http:}\OperatorTok{//}\NormalTok{www.w3.org}\OperatorTok{/}\DecValTok{2000}\OperatorTok{/}\DecValTok{01}\OperatorTok{/}\NormalTok{rdf}\OperatorTok{{-}}\NormalTok{schema\#}\OperatorTok{\textgreater{}}

    \KeywordTok{SELECT} \KeywordTok{DISTINCT}\NormalTok{ ?lab ?}\FunctionTok{abs}
    \KeywordTok{WHERE}\NormalTok{ \{}
\NormalTok{            \{ ?uri dbo}\CharTok{:country}\NormalTok{ res}\CharTok{:England}\NormalTok{ \} }\KeywordTok{UNION}
\NormalTok{            \{ ?uri dbo}\CharTok{:country}\NormalTok{ res}\CharTok{:United\_Kingdom}\NormalTok{ \} }\KeywordTok{UNION}
\NormalTok{            \{ ?uri dbo}\CharTok{:country}\NormalTok{ res}\CharTok{:Scotland}\NormalTok{ \} }\KeywordTok{UNION}
\NormalTok{            \{ ?uri dbo}\CharTok{:country}\NormalTok{ res}\CharTok{:Wales}\NormalTok{ \} }\KeywordTok{UNION}
\NormalTok{            \{ ?uri dbo}\CharTok{:location}\NormalTok{ res}\CharTok{:England}\NormalTok{ \} }\KeywordTok{UNION}
\NormalTok{            \{ ?uri dbo}\CharTok{:location}\NormalTok{ res}\CharTok{:United\_Kingdom}\NormalTok{ \} }\KeywordTok{UNION}
\NormalTok{            \{ ?uri dbo}\CharTok{:location}\NormalTok{ res}\CharTok{:Scotland}\NormalTok{ \} }\KeywordTok{UNION}
\NormalTok{            \{ ?uri dbo}\CharTok{:location}\NormalTok{ res}\CharTok{:Wales}\NormalTok{ \} .}

\NormalTok{            \{ ?uri rdf}\CharTok{:type}\NormalTok{ dbo}\CharTok{:Place}\NormalTok{ \} }\KeywordTok{UNION}
\NormalTok{            \{ ?uri rdf}\CharTok{:type}\NormalTok{ dbo}\CharTok{:Organisation}\NormalTok{ \} .}

\NormalTok{              ?uri rdfs}\CharTok{:lab}\NormalTok{ ?lab . }\KeywordTok{FILTER}\NormalTok{ (lang(?lab) }\OperatorTok{=} \StringTok{\textquotesingle{}en\textquotesingle{}}\NormalTok{)}
\NormalTok{              ?uri dbo}\CharTok{:abstract}\NormalTok{ ?}\FunctionTok{abs}\NormalTok{ . }\KeywordTok{FILTER}\NormalTok{ (lang(?}\FunctionTok{abs}\NormalTok{) }\OperatorTok{=} \StringTok{\textquotesingle{}en\textquotesingle{}}\NormalTok{)}
\NormalTok{    \} }\KeywordTok{LIMIT} \DecValTok{10000}
\end{Highlighting}
\end{Shaded}

To overcome the 10,000 query limit, the query was looped over several
times with a 10,000 query offset to obtain a full set of results.

\hypertarget{including-metadata}{%
\subsection{Including Metadata}\label{including-metadata}}

To obtain the further geographic information required for building the
hierarchical links between the extracted locations, the
\href{https://data.ordnancesurvey.co.uk/datasets/os-linked-data}{Ordnance
Survey Linked Data} was considered as it was accessible through a SPARQL
endpoint. However, the data itself did not contain the hierarchical
links required, and given such a large amount of data was being
extracted, it was not an efficient way to gather the data.
Alternatively, Ordnance Survey provides access to two core datasets
through the Edina Digimap service under the
\href{https://digimap.edina.ac.uk/webhelp/os/copyright/digimap_os_eula.pdf}{Educational
User Licence}, allowing for free unlimited access of the data for
\emph{Educational Use}. However, as this project is undertaken in
collaboration with Ordnance Survey, it is likely that the licence will
be more flexible.

The two datasets accessed through Digimap were Ordnance Survey Points of
Interest (POI), and OS Open Names. The POI data contains various
information regarding certain locations in the United Kingdom classified
as POI. A POI is generally defined as a location that a person may find
useful or interesting. In the case of this data, POI include locations
such as \emph{All Saints Church Hall}, the word \emph{Church}, or the
names of shops e.g.~\emph{The Co-operative}. While much of this data
does not provide useful locational information for this project, for any
unique named location it may provide additional meta information. The
POI dataset was linked with the names of DBpedia locations and provided
the additional metadata for the feature geometries \texttt{X},
\texttt{Y}, \texttt{admin\_boundary}, \texttt{geographic\_county}, and
\texttt{postcode}.

The OS Open Names dataset provided additional information for the
majority of DBpedia locations, linking was made particularly easy given
the presence of a variable in this dataset called
\texttt{SAME\_AS\_DBPEDIA} which enabled accurate linking with the
DBpedia dataset for the majority of locations. Additional locations were
linked by name as with the POI information. The OS Open Names data
provides the additional metadata \texttt{TYPE}; including
\emph{hydrography}, \emph{populatedPlace}, \emph{transportNetwork}, etc.
With a more granular \texttt{LOCAL\_TYPE.} Additionally
\texttt{feature\_easting} and \texttt{feature\_northing} geometries,
\texttt{POSTCODE}, \texttt{BOROUGH}, \texttt{COUNTY} and
\texttt{COUNTRY}, provided the in depth hierarchical structure as
outlined in \textbf{Figure \ref{f:hier}}.

\hypertarget{building-the-database}{%
\subsection{Building the Database}\label{building-the-database}}

Neo4j is the most popular graph database management system and has been
used in previous work relating to the construction of databases for use
in geographic natural language processing to semantically link
associations between places (Chen \emph{et al.},
\protect\hyperlink{ref-chen2018}{2018}\protect\hyperlink{ref-chen2018}{b},
\protect\hyperlink{ref-chen2018a}{2018}\protect\hyperlink{ref-chen2018a}{a};
Kim \emph{et al.}, \protect\hyperlink{ref-kim2017}{2017}).

First constraints were created to ensure no duplicates are created, and
to improve the efficiency of the database construction.

\vspace{2mm}

\begin{Shaded}
\begin{Highlighting}[]
\KeywordTok{CREATE} \KeywordTok{CONSTRAINT} \KeywordTok{ON}\NormalTok{ (pc}\CharTok{:Postcode}\NormalTok{) ASSERT pc.name }\KeywordTok{IS} \KeywordTok{UNIQUE}\NormalTok{;}
\KeywordTok{CREATE} \KeywordTok{CONSTRAINT} \KeywordTok{ON}\NormalTok{ (b}\CharTok{:Borough}\NormalTok{) ASSERT b.name }\KeywordTok{IS} \KeywordTok{UNIQUE}\NormalTok{;}
\KeywordTok{CREATE} \KeywordTok{CONSTRAINT} \KeywordTok{ON}\NormalTok{ (c}\CharTok{:County}\NormalTok{) ASSERT c.name }\KeywordTok{IS} \KeywordTok{UNIQUE}\NormalTok{;}
\KeywordTok{CREATE} \KeywordTok{CONSTRAINT} \KeywordTok{ON}\NormalTok{ (cy}\CharTok{:Country}\NormalTok{) ASSERT cy.name }\KeywordTok{IS} \KeywordTok{UNIQUE}\NormalTok{;}
\end{Highlighting}
\end{Shaded}

Following this, the complete data including DBpedia descriptions, and OS
metadata was read in periodically 10,000 rows at a time to ensure memory
constraints were followed. This code snippet shows the creation of the
\emph{Location} node, which includes the \texttt{name}, \texttt{type},
\texttt{X}, \texttt{Y}, and \texttt{abstract} variables. This was
repeated for each of the hierarchical locations contained within the
data.

\vspace{2mm}

\begin{Shaded}
\begin{Highlighting}[]
\KeywordTok{USING}\NormalTok{ PERIODIC }\KeywordTok{COMMIT} \DecValTok{10000}
\NormalTok{LOAD CSV }\KeywordTok{WITH}\NormalTok{ HEADERS }\KeywordTok{FROM} \OtherTok{"file:///wiki.csv"} \KeywordTok{AS} \KeywordTok{row}
\KeywordTok{WITH} \KeywordTok{row} \KeywordTok{WHERE} \KeywordTok{NOT} \KeywordTok{row}\NormalTok{.}\KeywordTok{label} \KeywordTok{IS} \KeywordTok{null}
\KeywordTok{MERGE}\NormalTok{ (l}\CharTok{:Location}\NormalTok{ \{name: }\KeywordTok{row}\NormalTok{.}\KeywordTok{label}\NormalTok{,}
                   \KeywordTok{type}\NormalTok{: }\KeywordTok{row}\NormalTok{.}\KeywordTok{TYPE}\NormalTok{,}
\NormalTok{                   X: }\KeywordTok{row}\NormalTok{.GEOMETRY\_X,}
\NormalTok{                   Y: }\KeywordTok{row}\NormalTok{.GEOMETRY\_Y,}
\NormalTok{                   abstract: }\KeywordTok{row}\NormalTok{.}\FunctionTok{abs}\NormalTok{\})}
\KeywordTok{RETURN} \FunctionTok{count}\NormalTok{(l);}

\OperatorTok{..}
\end{Highlighting}
\end{Shaded}

Creation of the relations between locations and their hierarchical
relations is given below, in this case postcodes containing locations:

\vspace{2mm}

\begin{Shaded}
\begin{Highlighting}[]
\KeywordTok{USING}\NormalTok{ PERIODIC }\KeywordTok{COMMIT} \DecValTok{10000}
\NormalTok{LOAD CSV }\KeywordTok{WITH}\NormalTok{ HEADERS }\KeywordTok{FROM} \OtherTok{"file:///wiki.csv"} \KeywordTok{AS} \KeywordTok{row}
\KeywordTok{WITH} \KeywordTok{row}\NormalTok{.}\KeywordTok{label} \KeywordTok{as}\NormalTok{ lname, }\KeywordTok{row}\NormalTok{.POSTCODE\_DISTRICT }\KeywordTok{AS}\NormalTok{ pcname}
\NormalTok{MATCH (l}\CharTok{:Location}\NormalTok{ \{name: lname\})}
\NormalTok{MATCH (pc}\CharTok{:Postcode}\NormalTok{ \{name: pcname\})}
\KeywordTok{MERGE}\NormalTok{ (pc)}\OperatorTok{{-}}\NormalTok{[rel}\CharTok{:CONTAINS}\NormalTok{]}\OperatorTok{{-}\textgreater{}}\NormalTok{(l)}
\KeywordTok{RETURN} \FunctionTok{count}\NormalTok{(rel);}

\OperatorTok{..}
\end{Highlighting}
\end{Shaded}

\hypertarget{results}{%
\section{Results}\label{results}}

Of the 49,923 \emph{Place} and \emph{Organisation} labelled results
extracted from the United Kingdom (excluding Northern Ireland), 23,020
were linked successfully with the OS POI, and OS Open Places data.

The Ordnance Survey POI data contains a total of 4,320,574 results, of
which, 70,959 were initially joined based on the DBpedia label. Of
these, only 6,646 provided unique abstracts, with analysis revealing
that DBpedia places with many repeated entries in the POI inventory
included general concepts such as \emph{Guide Post}, and company names
like \emph{Premier Inn}. Any ambiguous entry was therefore removed to
ensure the abstract was correctly associated with the POI. Following
this, the POI inventory provided a total of 5,534 unambiguous results.
OS Open Data provided 2,927,487 places, joining using the place name
first provided 29,367 results, and following the removal of duplicates,
this was reduced down to 13,881. It is noted that place names from
DBpedia which include additional information e.g.~\emph{Place Name,
(County)} are unlikely to be included through this method. Using the
\texttt{SAME\_AS\_DBPEDIA} column enabled the linking of a further
15,122 rows, assumed to be accurate.

\hypertarget{discussion-and-further-work}{%
\section{Discussion and Further
Work}\label{discussion-and-further-work}}

Previous studies have made use of the various classes DBpedia provides,
for example Gao \emph{et al.} (\protect\hyperlink{ref-gao2013}{2013})
used the \texttt{dbpedia-owl:nearestCity} relation in the \texttt{City}
class to obtain \emph{platial buffers} for city boundaries. They also
note the ability to perform \emph{platial joins} to obtain total
populations of all towns in the Californian County of Santa Barbara by
using the \texttt{partOf} predicate in relation to the \texttt{County}
class. Such examples however rely on both the comprehensive coverage of
classes and relations, and do not provide the granularity of relations
this study was hoping to achieve. It should be noted that it appears
that the United States often has more complete and consistent
representations on Wikipedia when compared with the United Kingdom.
Additionally, Rizzo \& Cano (\protect\hyperlink{ref-rizzo2015}{2015})
utilised the ontologies DBpedia provides for entity classification,
however, entity recognition work generally considers entities using
widely used entity tags, such as those present in the
\href{https://catalog.ldc.upenn.edu/LDC2013T19}{OntoNotes Release 5.0}.

Scheider \& Purves (\protect\hyperlink{ref-scheider2013}{2013}) discuss
the ability to combine Semantic Web reasoning with techniques associated
with Geographic Information Retrieval to localise places based on both
spatial and semantic relationships found in place descriptions. Namely
the relationships found between the place and other places, objects, or
activities. Building on this, the work presented in this report could be
used for the construction of a semantic gazetteer, first described by
Montello \emph{et al.} (\protect\hyperlink{ref-montello2003}{2003}),
with the goal of providing relevant geo-information, extracted from
descriptions, in a structured format. Early spatial cognition research
identified the semantic associations with place, Lynch
(\protect\hyperlink{ref-lynch1960}{1960}) for example describes the
cognitive concepts that are apparent from sketches of a city, while
Scheider \& Purves (\protect\hyperlink{ref-scheider2013}{2013}) note
that these concepts are also present within verbal place descriptions,
together with the spatial prepositions which link them.

The additional information provided through a semantic gazetteer is
considered essential for the ability to both extract and geocode spatial
expressions in natural language and to further improve toponym
disambiguation techniques. Chen \emph{et al.}
(\protect\hyperlink{ref-chen2018}{2018}\protect\hyperlink{ref-chen2018}{b})
demonstrate the use of graph databases to provide semantic links between
spatial relations from place descriptions, where places are nodes, and
spatial relationships are edges. Chen \emph{et al.}
(\protect\hyperlink{ref-chen2018a}{2018}\protect\hyperlink{ref-chen2018a}{a})
note however that by constructing place graphs in such a way, the
triples formed do not contain the majority of the additional context.
Particularly they note that these place graph models do not consider the
additional semantics or related human activities which would prove
useful for further spatial reasoning tasks.

Wolter \& Yousaf (\protect\hyperlink{ref-wolter2018}{2018}) describe in
detail the ability to derive information from place descriptions. When
presented with a river, humans may perceive the ability for it to be
followed, or with a hill, a person may describe going up or down,
information which may be present in descriptive language. Spatial
relations within descriptions rely on the reference frame of the place,
e.g.~\emph{``in front of''} is interpreted differently for a building
than for then end of a route, \emph{nearness} also relies on a
understanding of the scale of the place being considered. Human
cognitive principles also shape how place is described, however as often
is the case, Wolter \& Yousaf (\protect\hyperlink{ref-wolter2018}{2018})
note that place description and the place itself are likely all that is
known. Finally, Wolter \& Yousaf
(\protect\hyperlink{ref-wolter2018}{2018}) note that the language used
within a description may provide an indication of the granularity being
considered, and further entities being described will also provide
further information regarding the place.

Chen \emph{et al.}
(\protect\hyperlink{ref-chen2018a}{2018}\protect\hyperlink{ref-chen2018a}{a})
take the principals as described by Wolter \& Yousaf
(\protect\hyperlink{ref-wolter2018}{2018}), and use them to further
extend the place graph database model, taking all the information
provided by place descriptions to form detailed nodes containing
detailed semantic information regarding a place. Future work may mirror
and build on these concepts, enabled through the data provided in this
report, and build towards the construction of a semantic gazetteer.

\hypertarget{references}{%
\section{References}\label{references}}

\small

\hypertarget{refs}{}
\begin{cslreferences}
\leavevmode\hypertarget{ref-adams2013}{}%
\textbf{Adams, B. \& McKenzie, G.} (\textbf{2013}) 'Inferring thematic
places from spatially referenced natural language descriptions',
\emph{Crowdsourcing geographic knowledge}. Springer, pp. 201--221.

\leavevmode\hypertarget{ref-al-olimat2019}{}%
\textbf{Al-Olimat, H.S., Shalin, V.L., Thirunarayan, K. \& Sain, J.P.}
(\textbf{2019}) 'Towards Geocoding Spatial Expressions (Vision Paper)',
\emph{Proceedings of the 27th ACM SIGSPATIAL International Conference on
Advances in Geographic Information Systems - SIGSPATIAL '19}. Chicago,
IL, USA: ACM Press, 2019, pp. 75--78.

\leavevmode\hypertarget{ref-andogah2010}{}%
\textbf{Andogah, G.} (\textbf{2010}) \emph{Geographically Constrained
Information Retrieval}. p. 205.

\leavevmode\hypertarget{ref-chen2018a}{}%
\textbf{Chen, H., Vasardani, M., Winter, S. \& Tomko, M.}
(\textbf{2018a}) A Graph Database Model for Knowledge Extracted from
Place Descriptions. \emph{ISPRS International Journal of
Geo-Information}. 7 (6). p. 221.

\leavevmode\hypertarget{ref-chen2018}{}%
\textbf{Chen, H., Winter, S. \& Vasardani, M.} (\textbf{2018b})
Georeferencing places from collective human descriptions using place
graphs. \emph{Journal of Spatial Information Science}. (17). pp. 31--62.

\leavevmode\hypertarget{ref-gao2013}{}%
\textbf{Gao, S., Janowicz, K., McKenzie, G. \& Li, L.} (\textbf{2013})
\emph{Towards Platial Joins and Buffers in Place-Based GIS}. p. 8.

\leavevmode\hypertarget{ref-gey2006}{}%
\textbf{Gey, F., Larson, R., Sanderson, M., Bischoff, K., Mandl, T.,
Womser-Hacker, C., Santos, D. \& Rocha, P.} (\textbf{2006})
\emph{GeoCLEF 2006: The CLEF 2006 Cross-Language Geographic Information
Retrieval Track Overview}. p. 20.

\leavevmode\hypertarget{ref-gritta2019}{}%
\textbf{Gritta, M.} (\textbf{2019}) \emph{Where are you talking about?}
p. 159.

\leavevmode\hypertarget{ref-gritta2018}{}%
\textbf{Gritta, M., Pilehvar, M.T. \& Collier, N.} (\textbf{2018})
'Which Melbourne? Augmenting Geocoding with Maps', \emph{Proceedings of
the 56th Annual Meeting of the Association for Computational Linguistics
(Volume 1: Long Papers)}. Melbourne, Australia: Association for
Computational Linguistics, 2018, pp. 1285--1296.

\leavevmode\hypertarget{ref-ju2016}{}%
\textbf{Ju, Y., Adams, B., Janowicz, K., Hu, Y., Yan, B. \& McKenzie,
G.} (\textbf{2016}) 'Things and strings: Improving place name
disambiguation from short texts by combining entity co-occurrence with
topic modeling', \emph{European Knowledge Acquisition Workshop}.
Springer, 2016, pp. 353--367.

\leavevmode\hypertarget{ref-karimzadeh2019}{}%
\textbf{Karimzadeh, M., Pezanowski, S., MacEachren, A.M. \& Wallgrün,
J.O.} (\textbf{2019}) GeoTxt: A scalable geoparsing system for
unstructured text geolocation. \emph{Transactions in GIS}. 23 (1). pp.
118--136.

\leavevmode\hypertarget{ref-kim2017}{}%
\textbf{Kim, J., Vasardani, M. \& Winter, S.} (\textbf{2017}) Landmark
Extraction from Web-Harvested Place Descriptions. \emph{KI - Künstliche
Intelligenz}. 31 (2). pp. 151--159.

\leavevmode\hypertarget{ref-leidner2007}{}%
\textbf{Leidner, J.L.} (\textbf{2007}) Toponym resolution in text:
Annotation, evaluation and applications of spatial grounding. \emph{ACM
SIGIR Forum}. 41 (2). pp. 124--126.

\leavevmode\hypertarget{ref-leidner2011}{}%
\textbf{Leidner, J.L. \& Lieberman, M.D.} (\textbf{2011}) Detecting
geographical references in the form of place names and associated
spatial natural language. \emph{SIGSPATIAL Special}. 3 (2). pp. 5--11.

\leavevmode\hypertarget{ref-li2014}{}%
\textbf{Li, C. \& Sun, A.} (\textbf{2014}) 'Fine-grained location
extraction from tweets with temporal awareness', \emph{Proceedings of
the 37th international ACM SIGIR conference on Research \& development
in information retrieval - SIGIR '14}. Gold Coast, Queensland,
Australia: ACM Press, 2014, pp. 43--52.

\leavevmode\hypertarget{ref-lynch1960}{}%
\textbf{Lynch, K.} (\textbf{1960}) OCLC: 255254857 \emph{The image of
the city}. Publication of the Joint Center for Urban Studies Nachdr.
Cambridge, Mass.: MIT PRESS.

\leavevmode\hypertarget{ref-middleton2016}{}%
\textbf{Middleton, S.E. \& Krivcovs, V.} (\textbf{2016}) Geoparsing and
Geosemantics for Social Media: Spatio-Temporal Grounding of Content
Propagating Rumours to support Trust and Veracity Analysis during
Breaking News. \emph{ACM Transactions on Information Systems}. p. 27.

\leavevmode\hypertarget{ref-moncla2014}{}%
\textbf{Moncla, L., Renteria-Agualimpia, W., Nogueras-Iso, J. \& Gaio,
M.} (\textbf{2014}) 'Geocoding for texts with fine-grain toponyms: An
experiment on a geoparsed hiking descriptions corpus', \emph{Proceedings
of the 22nd ACM SIGSPATIAL International Conference on Advances in
Geographic Information Systems - SIGSPATIAL '14}. Dallas, Texas: ACM
Press, 2014, pp. 183--192.

\leavevmode\hypertarget{ref-montello2003}{}%
\textbf{Montello, D.R., Goodchild, M.F., Gottsegen, J. \& Fohl, P.}
(\textbf{2003}) \emph{Where's Downtown?: Behavioral Methods for
Determining Referents of Vague Spatial Queries}. p. 20.

\leavevmode\hypertarget{ref-rizzo2015}{}%
\textbf{Rizzo, G. \& Cano, A.E.} (\textbf{2015}) \emph{Making Sense of
Microposts (\#Microposts2015) Named Entity rEcognition \& Linking
Challenge}. p. 10.

\leavevmode\hypertarget{ref-roberts2010}{}%
\textbf{Roberts, K.} (\textbf{2010}) \emph{Toponym Disambiguation Using
Events}. p. 6.

\leavevmode\hypertarget{ref-scheider2013}{}%
\textbf{Scheider, S. \& Purves, R.} (\textbf{2013}) 'Semantic place
localization from narratives', \emph{Proceedings of The First ACM
SIGSPATIAL International Workshop on Computational Models of Place -
COMP '13}. Orlando FL, USA: ACM Press, 2013, pp. 16--19.

\leavevmode\hypertarget{ref-speriosu2013}{}%
\textbf{Speriosu, M.} (\textbf{2013}) \emph{Methods and Applications of
Text-Driven Toponym Resolution with Indirect Supervision}. p. 173.

\leavevmode\hypertarget{ref-stock2013}{}%
\textbf{Stock, K., Pasley, R.C., Gardner, Z., Brindley, P., Morley, J.
\& Cialone, C.} (\textbf{2013}) 'Creating a corpus of geospatial natural
language', \emph{International Conference on Spatial Information
Theory}. Springer, 2013, pp. 279--298.

\leavevmode\hypertarget{ref-sun2019a}{}%
\textbf{Sun, K., Zhu, Y., Pan, P., Hou, Z., Wang, D., Li, W. \& Song,
J.} (\textbf{2019}) Geospatial data ontology: The semantic foundation of
geospatial data integration and sharing. \emph{Big Earth Data}. 3 (3).
pp. 269--296.

\leavevmode\hypertarget{ref-tobin2010}{}%
\textbf{Tobin, R., Grover, C., Byrne, K., Reid, J. \& Walsh, J.}
(\textbf{2010}) 'Evaluation of georeferencing', \emph{Proceedings of the
6th Workshop on Geographic Information Retrieval - GIR '10}. Zurich,
Switzerland: ACM Press, 2010, p. 1.

\leavevmode\hypertarget{ref-wallgrun2018}{}%
\textbf{Wallgrün, J.O., Karimzadeh, M., MacEachren, A.M. \& Pezanowski,
S.} (\textbf{2018}) GeoCorpora: Building a corpus to test and train
microblog geoparsers. \emph{International Journal of Geographical
Information Science}. 32 (1). pp. 1--29.

\leavevmode\hypertarget{ref-weissenbacher2019}{}%
\textbf{Weissenbacher, D., Magge, A., O'Connor, K., Scotch, M. \&
Gonzalez-Hernandez, G.} (\textbf{2019}) 'SemEval-2019 Task 12: Toponym
Resolution in Scientific Papers', \emph{Proceedings of the 13th
International Workshop on Semantic Evaluation}. Minneapolis, Minnesota,
USA: Association for Computational Linguistics, 2019, pp. 907--916.

\leavevmode\hypertarget{ref-weissenbacher2015}{}%
\textbf{Weissenbacher, D., Tahsin, T., Beard, R., Figaro, M., Rivera,
R., Scotch, M. \& Gonzalez, G.} (\textbf{2015}) Knowledge-driven
geospatial location resolution for phylogeographic models of virus
migration. \emph{Bioinformatics}. 31 (12). pp. i348--i356.

\leavevmode\hypertarget{ref-wolter2018}{}%
\textbf{Wolter, D. \& Yousaf, M.} (\textbf{2018}) 'Context and Vagueness
in Automated Interpretation of Place Description: A Computational
Model', P. Fogliaroni, A. Ballatore, \& E. Clementini (eds.).
\emph{Proceedings of Workshops and Posters at the 13th International
Conference on Spatial Information Theory (COSIT 2017)}. Cham: Springer
International Publishing, pp. 137--142.

\leavevmode\hypertarget{ref-zhang2014}{}%
\textbf{Zhang, W. \& Gelernter, J.} (\textbf{2014}) Geocoding location
expressions in Twitter messages: A preference learning method.
\emph{Journal of Spatial Information Science}. (9). pp. 37--70.
\end{cslreferences}




\end{document}

\makeatletter
\def\@maketitle{%
  \newpage
%  \null
%  \vskip 2em%
%  \begin{center}%
  \let \footnote \thanks
    {\fontsize{18}{20}\selectfont\raggedright  \setlength{\parindent}{0pt} \@title \par}%
}
%\fi
\makeatother
